\documentclass{article}

\usepackage{booktabs,soul,tabu,pdfpages}		
\usepackage{enumerate}
\usepackage{tabularx}
\newcolumntype{C}{>{\centering\arraybackslash}X}
\newcolumntype{K}{>{\centering\arraybackslash}X}
%\usepackage{bm}
\usepackage[english]{babel}
\usepackage{lmodern}
\usepackage[utf8]{inputenc}
\usepackage[T1]{fontenc}
\usepackage{graphicx}
\usepackage[round]{natbib}
\usepackage{hyperref}					
\usepackage{booktabs,tabularx}
\usepackage{color,enumerate}
\usepackage{caption}
\usepackage{amsmath,amstext,amssymb} 
\usepackage{tabu,booktabs,color,pdfpages}
\usepackage{setspace}

\title{We Need It Now: \\
How To Analyze COVID Data Quickly}\\
\author{Ahmed Hifaz \\
	\small hifaz.ahmed@stud.hs-fresenius.de \\
	\small +49 151 660 77919 \\
	\small OH: 11:00-16:00 \\
	\and 
	Alisa Ishikawa \\
	\small ishikawa.alisa@stud.hs-fresenius.de \\
	\small +49 174 320 6012 \\
	\small OH: 10:30-14:30 \\
	\and 
	Pirneeta Bhambhani \\
	\small bhambhani.pirneeta@stud.hs-fresenius.de \\
	\small +49 157 332 68112 \\
	\small OH: 10:00-15:00 \\
}
\date{HS Fresenius, Cologne Campus \\
        January 5, 2020 15:00-16:30}

\begin{document}
	\maketitle
	
\begin{abstract}
\noindent
R is an open-source programming language that helps to analyze the spread and  consequences of viruses such as COVID-19. In particular, epidemiologists that seek to give policy advice employ R to combine data from different sources, to create various statistics, and to visualize the degree of the outbreak. This lecture will cover how epidemiologists utilize R as a tool in assessing viral outbreaks, how publicly available data can be structured and utilized in making analyses, and how to utilize visualization tools on R to create graphs to assist with conceptualizing, forecasting and understanding the characteristics of a virus and its degree of infectiousness. To visualize and analyze and the outbreak quickly is essential to guide political responses. The lecture also includes several real life case studies of COVID-19 to aid with the theoretical topics.
\end{abstract}
\subsection*{Learning Objectives}
	After this course you should be able to:\label{book}
\begin{itemize}
	\item Understand the importance of R in large-scale data analysis
    \item Accumulate data from desired authorised sources to produce a customized analysis
    \item Explore alternative resources for COVID-19 data acquisition
    \item Use R to visualize COVID-19 data with graphs and maps
\end{itemize}
\subsection*{Course Outline}
\begin{spacing}{1.0}
\subsubsection*{1 Why R is a significant tool of choice for epidemiologists}
    1.1 Data analytics and Epidemiology \\
    1.2 Role of 'r' number's in COVID-19 data collection \\
    1.3 What is RECON and what do they do? \\
    1.4 Web Scraping and Data Coalition \\
    1.5 Case Study: Saxony, Germany
\subsubsection*{2 How we can acquire COVID-19 data, just like epidemiologists}
    2.1 Gathering COVID-19 related data through authentic sources \\
    2.2 Strengths of R as a statistical application for forecasting/predictions
    \subsubsection*{3 How to support policy making with visualizations of COVID-19 data}
    3.1 Overview of plot() and ggplot() \\
    3.2 Case Study 1: Visualization capabilities using plot() \\
    3.3 Case Study 2: Visualization capabilities using ggplot()
\end{spacing}
\noindent 
\subsection*{Recommended Readings}

Ponce, M., & Sandhel, A. (2020). covid19.analytics: An R Package to Obtain, \indent Analyze and Visualize Data from the Corona Virus Disease Pandemic, 1–47. \indent Retrieved from https://arxiv.org/pdf/2009.01091.pdf

\noindent
\subsection*{Literature}
xxx

\subsection*{Resources for lecture notes, in-class slides and exercises}
Ishikawa, A., 2020. \textit{Aqualisa/HSF-MIBM2-COVID19}. [online] GitHub.  \\
\indent Available at: <https://github.com/aqualisa/HSF-MIBM2-COVID1  \\
\indent 9>.  \\
\noindent

\subsection*{Further Reading}

\end{document}